%%%%%%%%%%%%%%%%%%%%%%%%%%%%%%%%%%%%%%%%%
% Beamer Presentation
% LaTeX Template
% Version 1.0 (10/11/12)
%
% This template has been downloaded from:
% http://www.LaTeXTemplates.com
%
% License:
% CC BY-NC-SA 3.0 (http://creativecommons.org/licenses/by-nc-sa/3.0/)
%
%%%%%%%%%%%%%%%%%%%%%%%%%%%%%%%%%%%%%%%%%

%----------------------------------------------------------------------------------------
%	PACKAGES AND THEMES
%----------------------------------------------------------------------------------------

\documentclass[handout, 10 pt]{beamer}

\mode<presentation> {

% The Beamer class comes with a number of default slide themes
% which change the colors and layouts of slides. Below this is a list
% of all the themes, uncomment each in turn to see what they look like.

%\usetheme{default}
%\usetheme{AnnArbor}
%\usetheme{Antibes}
%\usetheme{Bergen}
%\usetheme{Berkeley}
% \usetheme{Berlin}
%\usetheme{Boadilla}
%\usetheme{CambridgeUS}
%\usetheme{Copenhagen}
%\usetheme{Darmstadt}
%\usetheme{Dresden}
%\usetheme{Frankfurt}
%\usetheme{Goettingen}
%\usetheme{Hannover}
%\usetheme{Ilmenau}
%\usetheme{JuanLesPins}
%\usetheme{Luebeck}
%\usetheme{Madrid}
%\usetheme{Malmoe}
%\usetheme{Marburg}
%\usetheme{Montpellier}
%\usetheme{PaloAlto}
%\usetheme{Pittsburgh}
%\usetheme{Rochester}
%\usetheme{Singapore}
%\usetheme{Szeged}
\usetheme{Warsaw}

% As well as themes, the Beamer class has a number of color themes
% for any slide theme. Uncomment each of these in turn to see how it
% changes the colors of your current slide theme.

%\usecolortheme{albatross}
%\usecolortheme{beaver}
%\usecolortheme{beetle}
%\usecolortheme{crane}
%\usecolortheme{dolphin}
%\usecolortheme{dove}
%\usecolortheme{fly}
%\usecolortheme{lily}
% \usecolortheme{orchid}
% \usecolortheme{rose}
%\usecolortheme{seagull}
%\usecolortheme{seahorse}
%\usecolortheme{whale}
%\usecolortheme{wolverine}

%\newcommand*{\myfont}{\fontfamily{tahoma}\selectfont}

% \setbeamertemplate{footline} % To remove the footer line in all slides uncomment this line
% \setbeamertemplate{footline}[page number] % To replace the footer line in all slides with a simple slide count uncomment this line

% \setbeamertemplate{navigation symbols}{} % To remove the navigation symbols from the bottom of all slides uncomment this line
}
\usepackage{graphicx,amssymb,amsmath, mathtools, outlines}
\usepackage{mathrsfs}
% \usepackage{tikz, standalone, float, comment, tikz-cd}
\usepackage{amscd, amsthm, amsfonts, braket, inputenc,xcolor,setspace, hyperref,comment, gensymb}


% \usepackage{subcaption}
% \usepackage{graphicx}
% \usepackage{enumerate}
% \def\VR{\kern-\arraycolsep\strut\vrule &\kern-\arraycolsep}
% \def\vr{\kern-\arraycolsep & \kern-\arraycolsep}
% \usepackage{booktabs} % Allows the use of \toprule, \midrule and \bottomrule in tables


% \usepackage{tikz-cd}
% \usepackage{extarrows}

% \pgfdeclarelayer{edgelayer}
% \pgfdeclarelayer{nodelayer}
% \pgfdeclarelayer{background}
% \pgfdeclarelayer{foreground}
% \pgfsetlayers{background,foreground,edgelayer,nodelayer,main}

\usepackage{blkarray}

\newcommand{\icol}[1]{% inline column vector
  \left(\begin{smallmatrix}#1\end{smallmatrix}\right)%
}

%%
%\newtheorem{theorem}{Theorem}

\newtheorem{prop}{Proposition}
\newtheorem{lemm}{Lemma}
\newtheorem{corol}{Corollary}
%\newtheorem{remark}{Remark}
%\newtheorem{definition}{Definition}
% \newcommand{\huptwo}{\textrm{H}^{2}(\Gamma,\textrm{Hom}_{\mathbb{F}_{p}}(V,V))}
% \newcommand{\charg}[3]{\textrm{H}^{#1}(#2,#3)}


%----------------------------------------------------------------------------------------
%	TITLE PAGE
%----------------------------------------------------------------------------------------

\title{Some Applications of Quantum Entanglement to Cryptography} % The short title appears at the bottom of every slide, the full title is only on the title page

\author{Wenqin Chen} % Your name
\institute[Smith College] % Your institution as it will appear on the bottom of every slide, may be shorthand to save space
{\normalsize{Smith College}\\ % Your institution for the title page
\smallskip
Thesis Advisors: Dr. David Meyer, Prof. Rajan Mehta
% Joint work with Killian Meehan\\
% \smallskip
% \large{BIRS-CMO}\\
}
\date{May 1, 2020} % Date, can be changed to a custom date

\begin{document}

\begin{frame}
\titlepage % Print the title page as the first slide
\end{frame}

\begin{frame}
\frametitle{Overview} % Table of contents slide, comment this block out to remove it
\tableofcontents % Throughout your presentation, if you choose to use \section{} and \subsection{} commands, these will automatically be printed on this slide as an overview of your presentation
\end{frame}

%----------------------------------------------------------------------------------------
%	PRESENTATION SLIDES
%----------------------------------------------------------------------------------------

%------------------------------------------------% Sections can be created in order to organize your presentation into discrete blocks, all sections and subsections are automatically printed in the table of contents as an overview of the talk
%------------------------------------------------

%\subsection{Subsection Example} % A subsection can be created just before a set of slides with a common theme to further break down your presentation into chunks

\section{Motivation}
\begin{comment}
The Austrian physicist Erwin Schrödinger proposed this thought experiment involving a cat in a sealed box, with a device of a 50\% chance of killing the cat in the next hour. In the classical world, the cat is either alive or dead after 1 hour. However, in the quantum physics, at the instance before the box was open, the cat is equally alive and dead at the same time. it's only when the box is open, we see a single definite state.

The atoms form a world of possibilities rather than one of definite things or facts.
\end{comment}
\begin{frame}{Schrödinger's cat}
\begin{figure}[h]
    \centering
    \includegraphics[scale=0.4]{"Schrodinger's cat".png}
    \caption{The cat is a blur of probability (half alive and half dead) until we observe it}
    \label{fig:Schrödinger's cat}
\end{figure}    
\end{frame}

\begin{comment}
Another shocking thing about quantum mechanics is such phenomenon described by Albert Einstein as the "spooky action at a distance". What's so spooky about it is that one event in the universe can instantaneously affect another event arbitrarily far away, which seems to imply faster than light communication. We now refer to as entanglement and will go over it in more detail later. 
\end{comment}
\begin{frame}{"Spooky action at a distance"}
\begin{figure}[h]
    \centering
    \includegraphics[scale=0.2]{"actual entanglement".jpg}
    \caption{The First-Ever Photo of Quantum Entanglement captured by physicists at the University of Glasgow.}
    \label{fig:actual entanglement}
\end{figure}
\end{frame}

\begin{comment}
In cryptography, Alice and Bob try to communicate securely under the prying eyes of Eve. In my thesis, I investigate to apply quantum mechanics phenomenon such as entanglement to enhance security in cryptography.
% So that's one motivation behind my thesis, to utilize entanglement to better secure message transmission from Alice to Bob under the prying eyes of Eve. This is particularly important considering that the current crypto systems such as RSA can be easily broken with quantum computers.
emphasize it's a collaborative game between Alice and Bob against Eve
\end{comment}

\begin{frame}{Alice and Bob}
\begin{figure}[h]
    \centering
    \includegraphics[scale=0.8]{"Alice Bob Eve".png}
    \label{fig:Alice Bob Eve}
\end{figure}    
\end{frame}

\section{Quantum Mechanics With Linear Algebra}

\begin{comment}
Mathematically, a {\emph{qubit}} is a unit vector in a two-dimensional complex vector space.  In the same way that the bit is the fundamental unit of classical computation and information, the quantum bit, or qubit is the basic unit of quantum information.  With this motivation in mind, we sometimes write $\ket{0} = \icol{1\\0}$, and $\ket{1} = \icol{0\\1}$. Note that the symbols "$\ket{0}$" and "$\ket{1}$" are read "ket one" and "ket two".\footnote{We are using so-called Bra-ket notation common in quantum mechanics}  While bits may only take the values $0$ or $1$, a qubit is a complex linear combination, or \textit{superposition} of $\ket{0}$ and $\ket{1}$ with norm equal to one. 
\end{comment}

\begin{frame}
\frametitle{Qubit}

\begin{itemize}
    \item Let $\ket{0} = \icol{1\\0}$, $\ket{1} = \icol{0\\1}$.
    \pause
    \item A \textbf{qubit} $\ket{\psi}$ is a two-dimensional "ket" vector, $\ket{\psi}=\alpha\ket{0}+\beta\ket{1}$, 
where $\alpha,\beta\in\mathbb{C}$, with $|\alpha|^2+|\beta|^2=1$.
\pause
\item i.e. a \textit{superposition} (complex linear combination) of $\ket{0}, \ket{1}$.
\pause
    \begin{figure}[h]
    \centering
    \includegraphics[scale=0.3]{"Qubit".png}
    \caption{One physical interpretation of $\ket{0},\ket{1}$: the "ground" state and the "exited" state of an electron}
    \label{fig: qubit}
    \end{figure}
\pause
\item More generally, a \textbf{qudit} is a unit vector in ${\mathbb{C}}^d$.
\end{itemize}
\end{frame}

%Same as inner product, except that it's conjugate symmetric.
\begin{frame}

\frametitle{Hermitian inner product}

If $V$ is a complex vector space, a Hermitian inner product $\langle\hspace{.1cm},\hspace{.1cm}\rangle$ is a function from $V \times V$ to $\mathbb{C}$ which satisfies: 
\pause
\begin{enumerate}
\item $\langle\ket{v}, \sum_i \lambda_i \ket{w_i}\rangle=\sum_i \lambda_i \langle\ket{v}, \ket{w_i}\rangle$
\pause
\item $\langle\ket{v}, \ket{w}\rangle=\overline{\langle\ket{w}, \ket{v}\rangle}$
\pause
\item $\langle\ket{v}, \ket{v}\rangle \ge 0$ \textrm{ with equality if and only if }$\ket{v}=0$
\pause
\end{enumerate}
for all $\ket{v}$ and $\ket{w}$.\footnote{Note that in the above, $\overline{z}$ denotes the conjugate of the complex number $z$.} 

\bigskip
For example, $\mathbb{C}^n$ is an inner product space.
\begin{equation} \label{inner product defn}
  \langle \ket{v},\ket{w}\rangle=\sum\limits_{i=1}^n \overline{v_i} w_i \textrm{, where }
  \ket{v}=\icol{v_1\\ \vdots\\v_n} \textrm{ and }\ket{w}=\icol{w_1\\ \vdots\\w_n}.
\end{equation}
\end{frame}

\begin{frame}{Adjoint and bra vector}
    If $L:V \to W$ is a linear transformation between two inner product spaces, the {\emph{adjoint of }}$L$, $L^\dagger$ is the linear transformation from $W$ to $V$ characterized by the property that, $\forall \ket{v} \in V, \forall \ket{w} \in W$,
\begin{equation}
    \langle\ket{w}, L\ket{v}\rangle=\langle L^\dagger \ket{w}, \ket{v}\rangle .
\end{equation}
\pause
\begin{itemize}
    % \item Notice LHS inner product is taken in W, while RHS inner product is taken in V.
    % \pause
    \item When $V={\mathbb{C}}^n, W={\mathbb{C}}^m$ and $L(\ket{v})=A\ket{v}$ for some matrix $A \in M_{m.n}(\mathbb{C})$, the adjoint of $L$ is left multiplication by the conjugate transpose of $A$. i.e. $A^\dagger=(\overline{A})^T$.
    \pause
    \item When the matrix $A$ is $\ket{v}$, $\ket{v}^\dagger$ is just the conjugate transpose of the $\ket{v}$. Write $\bra{v}={\ket{v}}^\dagger$, read "bra v".
    \pause
    \item
    \begin{equation}
    \bra{v}\cdot \ket{w}=\sum_i \overline{v_i}w_i=\langle \ket{v},\ket{w}\rangle,
    \end{equation}
    where the dot on the left-hand side is just matrix multiplication.
    \pause
    \item We will write $\langle \ket{v},\ket{w}\rangle$ as $\braket{v|w}$ and refer to the inner product as the "bra-ket" of v and w.
\end{itemize}
\end{frame}

\begin{comment}
Such is the motivation for the "bra-ket" notation, a convention in quantum mechanics.
\end{comment}
% \begin{frame}{One special case: Bra Vector}
% \begin{itemize}
%     \item When the matrix $A$ is a column vector $\ket{v}$, Then, clearly $\ket{v}^\dagger$ is just the conjugate transpose of the $\ket{v}$.
%     \pause
%     \item Write $\bra{v}={\ket{v}}^\dagger$, read "bra v".
%     \pause
%     \item Then
%     \begin{equation}
%     \bra{v}\cdot \ket{w}=\sum_i \overline{v_i}w_i=\langle \ket{v},\ket{w}\rangle,
%     \end{equation}
%     where the dot on the left-hand side is just matrix multiplication.
%     \pause
%     \item We will write $\langle \ket{v},\ket{w}\rangle$ as $\braket{v|w}$ and refer to the inner product as the "bra-ket" of v and w.
%     % Since the inner product agrees with matrix multiplication using this notation, as is customary we will write the inner product of $\ket{v}$ and $\ket{w}$ as $\braket{v|w}$, and refer to the inner product as the "bra-ket" of v and w.
% \end{itemize}
% \end{frame}

\begin{frame}
\frametitle{Tensor product}
\begin{itemize}
    \item The \textit{tensor product} is a way of {\emph{multiplying}} vector spaces to form larger vector spaces. 
    \pause
    \item The elements of $V \otimes W$ are linear combinations of "simple tensors" of the form $\ket{v} \otimes \ket{w}$, where $\ket{v} \in V, \ket{w} \in W$. 
    \pause
    \item If $\{\ket{a_1}, \ket{a_2}...\ket{a_m}\}, \{\ket{b_1},\ket{b_2}, ...\ket{b_n}\}$ are bases for V and W respectively, then the collection $\{\ket{a_i}\otimes\ket{b_j}\}$ is a basis for $V \otimes W$ for $1\leq i \leq m$ and $1 \leq j \leq n$.
\end{itemize}
\pause
\begin{block}{Example}
Given two vectors $\ket{\psi_1}=\icol{0\\1}$ and $\ket{\psi_2}=\icol{\frac{1}{\sqrt{3}}\\\frac{\sqrt{2}}{\sqrt{3}}}$, the tensor product is given by
\begin{equation}
\ket{\psi_1} \otimes\ket{\psi_2}
= \icol{0\\1} \otimes \icol{\frac{1}{\sqrt{3}}\\\frac{\sqrt{2}}{\sqrt{3}}} = \icol{ 0\ket{\psi_2} \\ 1\ket{\psi_2}}=\icol{0\\0\\\frac{1}{\sqrt{3}}\\\frac{\sqrt{2}}{\sqrt{3}}}.
\end{equation}
\end{block}
\end{frame}

\section{Measurements}
\begin{comment}
Note the following description is not mathematically rigorous.
\end{comment}

\begin{frame}{Four postulates of quantum mechanics}
    \begin{enumerate}
        \item Any isolated physical system is a complete inner product space and can be completely described by unit vectors in the space.
        \pause
        \item The evolution of a closed quantum system through time is described by a unitary transformation.
        \pause
        \item Next slide.
        \pause
        \item If we are in a composite quantum system made up of two or mode distinct physical system, the state of the composite system is described by the tensor product of the inner product spaces of the component physical systems. 
    \end{enumerate}
\end{frame}

\begin{comment}
Recall Schrodinger's cat.
\end{comment}
\begin{frame}{Postulate 3 of Quantum Mechanics: Measurements}
\begin{itemize}
    \item When {\emph{any}} external object (including measuring equipment) observes a quantum system, the quantum system is no longer closed.
    \pause
    \item Hence it's not necessarily subject to unitary evolution as in Postulate 2. 
    \pause
    \item Postulate 3 describes the effects of these measurements on a quantum system.
    \pause
    \item There are other types of measurements covered by the postulate, but we will only focus on measurement with respect to orthonormal basis.
\end{itemize}

% \pause
%     A projective measurement is described by an \textit{observable}, M, a Hermitian operator on the state space of the sytem being observed. The observable has a spectral decomposition,
%     \begin{equation}
%         M=\sum_m m P_m
%     \end{equation}
%     where $P_m$ is the projector onto the eigenspace of M with eigenvalue m. The possible outcomes of the measurement correspond to the eigenvalues, m, of the observable.
\end{frame}

% \begin{frame}{Frame Title}
%         A projective measurement is described by M, a Hermitian operator on the state space of the sytem being observed. The observable has a spectral decomposition,
%     \begin{equation}
%         M=\sum_m m P_m
%     \end{equation}
%     where $P_m$ is the projector onto the eigenspace of M with eigenvalue m. The possible outcomes of the measurement correspond to the eigenvalues, m, of the observable. 
    
%     \begin{itemize}
%         \item 
%     \end{itemize}
% \end{frame}


\begin{frame}{Projective measurement}
\begin{itemize}
    \item Quantum measurements are described by a collection $\{M_m\}$ of measurement operators which are Hermitian and square to themselves. i.e. $M_m^2=M_m$ and $M_m^\dagger=M_m$.
    \pause
    \item These operators act on the state space of the system being measured, and the index {\emph{m}} refers to the measurement outcomes that may occur in the experiment. 
    \pause
    \item If the quantum system is in state $\ket{\psi}$ immediately prior to measurement, then the probability that outcome {\emph{m}} occurs is given by 
    \begin{gather*}
       p(m)=\bra{\psi}M_m \ket{\psi},
    \end{gather*}
    in which case the state of the system after the measurement is
    \begin{gather*}
        \frac{M_m \ket{\psi}}{\sqrt{\bra{\psi}M_m \ket{\psi}}}.
    \end{gather*}
    \pause
    \item Lastly, the measurement operators are required to satisfy the completeness equation $\sum\limits_m M_m =I$.   
\end{itemize}
\end{frame}



\begin{frame}{Example: $\{M_m\}$ as an orthonormal basis }
 If $\{\ket{u_1}, \ket{u_2}, ...\ket{u_k}\}$ is an orthonormal basis for ${\mathbb{C}}^k$ and $\ket{v} \in {\mathbb{C}}^k$ then, 
$$\ket{v}=\sum\limits_i\braket{u_i|v}\ket{u_i} .$$
Then
\begin{eqnarray*}
    (\sum_i \ket{u_i}\bra{u_i})\ket{v}&=&
    \sum_i \ket{u_i}\bra{u_i}\sum_j \braket{u_j|v} \ket{u_j}\\
    &=& \sum_{i,j}\braket{u_j|v}\ket{u_i}\braket{u_i|u_j}\\
    &=& \sum_{i,j}\braket{u_j|v}\ket{u_i}\delta_{i,j}\\
    &=&\sum_i \braket{u_i|v}\ket{u_i}\\
    &=&\ket{v}.
\end{eqnarray*}

Since the above holds for any $\ket{v}$, $\sum\limits_i \ket{u_i}\bra{u_i}=I$.   
\end{frame}

\begin{frame}{Example (continued)}
\begin{itemize}
    \item Consider the orthonormal basis $\{\ket{u_m}\bra{u_m}\}$ as a family of measurement operators 
    \pause
    \item Squares to itself: 
    $\ket{u_m}\bra{u_m} \ket{u_m}\bra{u_m}=\ket{u_m}\bra{u_m}$
\end{itemize}
\end{frame}

% \begin{frame}{Example}
% \begin{block}{}
%     Consider a projective measurement described by the observable: \begin{equation}
%     Z=\ket{0}\bra{0}-\ket{1}\bra{1}
%     \end{equation}
%     and a quantum state to be measured:
%         $\ket{\psi}=\frac{\ket{0}+\ket{1}}{\sqrt{2}}$
% \end{block}
% \pause
% \begin{itemize}
%     \item $P_{+1}=\ket{0}\bra{0}$ is the projector onto the eigenspace of Z with eigenvalue 1.
%     \pause
%     \item $P_{-1}=\ket{1}\bra{1}$ is the projector onto the eigenspace of Z with eigenvalue -1.
%     \pause
%     \item Measurement of Z on $\ket{\psi}$ yields the outcome 1 with probability $\braket{\psi|0}\braket{0|\psi}=\frac{1}{2}$. Similarly, the outcome -1 has probability $\frac{1}{2}$.
% \end{itemize}
% \end{frame}

\begin{frame}{Example: Orthogonal states can be reliably distinguished}
\begin{block}{}
\begin{itemize}
    \item Suppose Alice chooses a state $\ket{\psi_i}$ from some fixed set of states known to both herself and to Bob.
    \pause
    \item She gives the state to Bob, whose task is to identify the index $i$ from the state Alice has given him by measuring.
\end{itemize}
\end{block}
\pause

If the collection of states $\{\ket{\psi_j}\}_{j=1}^n$ is orthonormal, then Bob can determine the index i correctly with certainty.
\pause
\bigskip

If Alice picked $\ket{\psi_i}$, the probability of observing $i$ and $j\neq i$ is given by:
\begin{align}
p(i)=\braket{\psi_i | M_i |\psi_i}=\braket{\psi_i |\psi_i}\braket{\psi_i |\psi_i}=1^2=1\\
p(j)=\braket{\psi_i | M_j |\psi_i}=\braket{\psi_i |\psi_j}\braket{\psi_j |\psi_i}=0^2=0
\end{align}
\end{frame}


\section{Classical Cryptography vs Quantum Cryptography}
% \begin{frame}{Goals of Cryptography: Correctness and Security}
% \begin{itemize}
% \item An encryption scheme (Enc, Dec) is correct if and only if for all possible messages m, and all possible keys k, we have m=Dec(k, Enc(k,m)).
% \smallskip
% \pause
% \item An encryption scheme (Enc, Dec) is \textit{secure} if and only if for all prior distributions p(m) over messages, and all messages m, we have
% \begin{equation}
%     p(m)=p(m|e),
% \end{equation}
% where $e=Enc(k,m)$, and $p(m|e)$ is the conditional probability of m given e.
% \end{itemize}
% \end{frame}

\begin{comment}
if alice and bob can agree upon a key that's as long as the length of the message, then in a sense it's unbreakable. not saying precisely about one-time pad, but show an example of the transcription of hello
\end{comment}
\begin{frame}{One-time pad with an example}
\begin{table}[h]
\centering 
\resizebox{\columnwidth}{!}{
\begin{tabular}{|c|c|c|c|c|c|}
\hline
character & h & e & l & l & o \\ \hline \pause
plaintext & 0110 1000 & 0110 0101 & 0110 1100 & 0110 1100 & 0110 1111 \\ \hline \pause
pad character & c & r & y & p & t \\ \hline \pause
one-time pad & 0110 0011 & 0111 0010 & 0111 1001 & 0111 0000 & 0111 0100 \\ \hline \pause
ciphertext & 0000 1011 & 0001 0111 & 0001 0101 & 0001 1100 & 0001 1011 \\ \hline
\end{tabular}}
\caption{Alice's encoding process from "hello" to the ciphertext represented by a string of bits}
\label{table: one-time pad encoding}
\end{table}
\pause

\begin{table}[h] 
\centering
\resizebox{\columnwidth}{!}{
\begin{tabular}{|c|c|c|c|c|c|}
\hline
ciphertext & 0000 1011 & 0001 0111 & 0001 0101 & 0001 1100 & 0001 1011 \\ \hline \pause
one-time pad & 0110 0011 & 0111 0010 & 0111 1001 & 0111 0000 & 0111 0100 \\ \hline \pause
plaintext & 0110 1000 & 0110 0101 & 0110 1100 & 0110 1100 & 0110 1111 \\ \hline \pause
plaintext character & h & e & l & l & o \\ \hline
\end{tabular}}
\caption{Bob's decipher process from the ciphertext back to "hello"}
\label{table: one-time pad decoding}
\end{table}
% \begin{itemize}
%     \item for a message $m \in \{0, 1\}^n$ using a key $k \in \{0, 1\}^n$
%     \pause
%     \item Encryption:
%     \begin{equation}
%     Enc(k,m)=(m \oplus k)=(m_1 \oplus k_1, m_2 \oplus k_2, \hdots, m_n \oplus k_n)=(e_1, \hdots, e_n)=e
%     \end{equation}
%     where $m_i \oplus k_i = m_i + k_i mod 2$.
%     \pause
%     \item Decryption:
%     \begin{equation}
% Dec(k, e)=e \oplus k = (e_1 \oplus k_1, e_2 \oplus k_2, \hdots, e_n \oplus k_n)
% \end{equation}
% \end{itemize}
\end{frame}

\begin{comment}
In many collaborative games, two players using a {\emph{quantum strategy}} can achieve a higher winning probability than if they use a {\emph{classical strategy}}. Because of this, many probabilistic games can be thought of as tests for entanglement.  This also illustrates why one might try to make use of quantum mechanics in cryptography.  We illustrate this phenomenon with an example of a guessing game.
\end{comment}

\begin{frame}{A game}
Alice and Bob work collaboratively to craft a strategy to maximize their team score.  Once they have agreed on a strategy, the game begins, at which point all their communication must cease.  The rules of the game are as follows:
\pause
\begin{enumerate}
\item Alice is presented with an element $x$ of the set $\{0,1\}$.  This element will serve as her {\emph{question}}.
\pause
\item Alice responds to her question by choosing $a$ an element of {\emph{another}} copy of the set $\{0,1\}$.  We will call $a$ Alice's {\emph{answer}}. \pause
\item Without seeing what has happened with Alice, Bob is also presented with an element $y$ of yet another copy of $\{0,1\}$. The element $y$ is Bob's question.
\pause
\item Bob too answers his question by choosing his answer $b$ from $\{0,1\}$.
\end{enumerate}
\pause
Once both answers are picked, we are ready do see if Alice and Bob have won.  They win the game if and only if their answers satisfy the equation
\begin{equation*}
 a+b =xy\hspace{.1cm}(mod 2) 
\end{equation*}
\end{frame}

\begin{frame}{Classical Strategy}
\begin{itemize}
    \item $f_A(x)=a$ and $f_B(y)=b$
    \begin{figure}[h]
    \centering
    \includegraphics[scale=0.3]{"classical strategy".png}
    \label{fig: classical strategy diagram}
    \end{figure}
    \item the maximum winning probability is $\frac{3}{4}$
\end{itemize}
\end{frame}

% still playing the same game, but now have the tool of entangled state and measuring on different bases
% use the picture and split it to 3 steps (entangled part->Alice->Bob)

\begin{comment}
this can be {\emph{improved}} by making use of quantum mechanics!  The idea is that Alice and Bob can make use of a strategy where even though neither sees which question the other is asked, their answers are {\emph{correlated}}.
\begin{equation*}
    \ket{\psi_{AB}}=\frac{1}{\sqrt{2}}(\ket{00}+\ket{11})
\end{equation*}


$$0.75<0.85$$
It's s simplified version of "CHSH experiment".
\end{comment}
\begin{frame}{Quantum Strategy:     $\ket{\psi_{AB}}=\frac{1}{\sqrt{2}}(\ket{00}+\ket{11})$}
    \begin{figure}[h]
    \centering
    \includegraphics[scale=0.42]{"quantum strategy diagram".png}
    \label{fig: quantum strategy diagram}
    \end{figure}
\end{frame}

% \begin{frame}{Implications}
% % Maybe explains why quantum is able to achieve better results - locality (don't mention the words).
% % This shows that quantum strategy can achieve better results.
% $$0.75<0.85$$
% \end{frame}

\section{Classifications of entanglement}
\begin{frame}{Rank of the Coefficient Matrix}
\begin{prop}
\label{rank prop}
Let $\ket{\psi}$ be a state in $\mathbb{C}^n \otimes \mathbb{C}^n$, and let $S=\{\ket{e_i}\}$ be the standard basis for $\mathbb{C}^n$. Define the coefficient matrix $M(\ket{\psi})$ by
\begin{equation}
M(\ket{\psi})=(a_{i,j}) \textrm{, where}\ket{\psi}=\sum\limits_{i,j}a_{i,j}|e_i e_j\rangle . 
\end{equation}
Also, set $S(\ket{\psi})=\{m\in \mathbb{N}: \ket{\psi}=\sum_{i=1}^m\ket{v_i w_i}, for \ket{v_i}, \ket{w_i} \in \mathbb{C}^n\}$.\\  Then, $rank(M(\ket{\psi}))=\textrm{min} (S(\ket{\psi})$.
\end{prop}
\pause

\begin{corol}
If $rank(M(\ket{\psi}))=1$, then the state $\ket{\psi}$ is separable. Otherwise, $\ket{\psi}$ is entangled.
\end{corol}
\end{frame}

\begin{frame}{Strength of Entanglement}
    \begin{prop}
\label{entanglement-rank}
Let $\ket{\psi} \in \mathbb{C}^n \otimes \mathbb{C}^{n}$ be a quantum state representing a composite system of Alice and Bob. Say Alice first makes a measurement and observes an outcome with end state $\ket{\varphi}\bra{\varphi}$. Let the state of Bob's system after Alice has performed the measurement be denoted by $\ket{\theta}\bra{\theta}$. Then,
\pause
\begin{enumerate}
    \item $rank(M(\ket{\psi}))=1$ if and only if $\ket{\theta}\bra{\theta}$ is a constant matrix that doesn't depend on $\ket{\varphi}$. In this situation, the state of Bob's system contains no information about Alice.
    \pause
    \item $rank(M(\ket{\psi}))=n$ if and only if $\ket{\psi}$ is maximally entangled. In this situation, the state of Bob's system contains perfect information about Alice.
    \pause
    \item If $1<rank(M(\ket{\psi}))<n$, then the state of Bob's system only contains some information about Alice.
\end{enumerate}
\end{prop}
\end{frame}

\begin{frame}{Post-measurement State Lemma}
\begin{lemm}
\label{end state lemma}
Let $\ket{\psi} \in \mathbb{C}^n \otimes \mathbb{C}^{n}$ be a quantum state representing a composite system of Alice and Bob.  The shared density matrix is therefore $\rho = \ket{\psi}\bra{\psi}$. Then,
\begin{enumerate}
\item Say Alice makes a measurement and observes $\ket{\varphi}\bra{\varphi}$, then Bob's post-measurement state is $\ket{\theta}\bra{\theta}$, where $\ket{\theta}=\frac{1}{\sqrt{C}}M(\ket{\psi})^T \cdot \overline{\ket{\varphi}}$, where $C$ is a constant.
\item Similarly, if Bob makes a measurement and observes observes $\ket{\varphi}\bra{\varphi}$, then Alice's post-measurement state is $\ket{\theta}\bra{\theta}$, where $\ket{\theta}=\frac{1}{\sqrt{C}}M(\ket{\psi})^T\cdot \overline{\ket{\varphi}}$, where $C$ is a constant.\\

\end{enumerate}
\end{lemm}
\end{frame}
\begin{frame}{Entanglement and Measuring}
    \begin{prop}
Let $\ket{\psi} \in \mathbb{C}^n \otimes \mathbb{C}^{n}$ be a quantum state representing a composite system of Alice and Bob. Say Alice makes a measurement with respect to a basis known to both herself and Bob.  If the transpose of the coefficient matrix preserves orthogonality, then Bob can determine Alice's measurement outcome with certainty. 
\end{prop}
\end{frame}
% %--------------------SLIDE 7____________-----------------------
% \begin{frame}
% \frametitle{Algebraic Stability}


% \pause

% \begin{block}{}
% \begin{theorem}[Isometry Theorem]
% Let $P = (0,\infty)$, ${\mathcal{T}}(P) = ([0,\infty),+)$.  Then the interleaving metric $D$ equals the bottleneck metric $D_B$.
% \end{theorem}
% \end{block}

% \pause
% \smallskip

% This suggests the following representation-theoretic analogue of the isometry theorem.
% \pause

% \smallskip

% Let $P$ be a finite poset and let $K$ be a field.  Choose a full subcategory  $\mathcal{C} \subseteq A(P)$-mod, and let 
% \begin{itemize}
% \pause
% \item $D$ be the interleaving metric restricted to $\mathcal{C}$, and
% \pause
% \item $D_B$ be a bottleneck metric on $\mathcal{C}$ which incorporates some algebraic information.
% \end{itemize}
% \pause
% Prove that $Id:(\mathcal{C},D) \to(\mathcal{C},D_B) $ is an isometry.
% \end{frame}


\begin{frame}{Links}
\begin{itemize}
    \item Overleaf link to this presentation: \href{https://www.overleaf.com/read/nwtrbjbjkznx}{https://www.overleaf.com/read/nwtrbjbjkznx}
    \item Overleaf link to this thesis:
    \href{https://www.overleaf.com/read/xxvfvqzvcfjx}{https://www.overleaf.com/read/xxvfvqzvcfjx}
    \item Github link to this presentation:
    \href{https://github.com/YoWenqin/Math-Thesis-Presentation}{https://github.com/YoWenqin/Math-Thesis-Presentation}
    \item Github link to this thesis:
    \href{https://github.com/YoWenqin/Wenqin-Chen-Math-Honor-Thesis-Reformated}{https://github.com/YoWenqin/Wenqin-Chen-Math-Honor-Thesis-Reformated}
    
\end{itemize}
\end{frame}

\begin{frame}


\begin{center}
\huge{THANK YOU!}

\end{center}

\end{frame}












%%%%%%%%%%%%%%%%%%%%%%%%%%%%





%%%%%%%%%%%%End ADDed%%%%%%%%%%%%%%%%%%%%%%%%%%

\end{document}